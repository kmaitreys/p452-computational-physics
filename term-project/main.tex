\documentclass{article}
\usepackage{graphicx} % Required for inserting images
\usepackage{amsmath} % Required for some math elements

\title{\textsc{Term Project}}
\title{\textbf{Synthetic Spectra for Interstellar
        Molecules in LTE and non-LTE regimes}}
\author{Maitrey Sharma}
\date{April 23, 2024}
\begin{document}

\maketitle
\begin{abstract}

\end{abstract}
\section{Introduction}
Consider a mono-atomic or singly-charged chemical species in gaseous state with
\(N\) energy levels defined as \(E_i\) with \(i = 1 \ldots N\) in ascending
order. Pertaining to the temperature this chemical species exists at, each
level will have certain occupation or population. These occupation numbers can
change due to collisions between neighboring molecules and if the environment
is sufficiently dense and temperatures are sufficiently high, this population
distribution can be deemed as thermal and can be expressed as
\begin{equation}
    \dfrac{N_j}{N_i} = {e}^{-(E_j - E_i)/k_B T}
\end{equation}
Here, \(N_i\) (and \(N_j\)) represents the level occupation number density of state
\(i\) (and \(j\)), \(k_B\) is the Boltzmann constant and \(T\) is the temperature of the
gas. This thermal distribution is basis for the condition of \textit{local thermodynamic
    equilibrium} or LTE.\@ In other words, the mean free path of the (excited) molecules
is much, much less than the scales at which the temperature varies in the medium.

\section{Radiative Transfer}
\subsection{The Escape Probability Scheme}
\subsection{Lambda Iteration and Accelerated Lambda Iteration}

\end{document}
